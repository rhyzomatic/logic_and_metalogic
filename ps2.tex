\documentclass[11pt]{article}
\usepackage{graphicx}
\usepackage[font=footnotesize, labelfont={sf,bf}, margin=1cm]{caption}
\usepackage{mathtools}
\usepackage{wrapfig}
\usepackage[letterpaper, portrait, margin=1in]{geometry}
\usepackage{framed}
\usepackage{listings}
\usepackage{color}
\usepackage{float}
\usepackage[utf8]{inputenc}
\usepackage{enumerate}
\usepackage[titletoc,toc,title]{appendix}
\usepackage{amsfonts}
\usepackage{amssymb}
\newcommand{\?}{\stackrel{?}{=}}
%Gummi|065|=)
\title{\textbf{Problem Set 2}}
\author{}
\date{}
\begin{document}

\maketitle

If the tape is not completely empty, then there must be at least one symbol on the tape and that symbol must be a finite number of squares away from wherever the head begins. This is because the head could never start infinitely far away from that symbol.

We can then write a machine which on each iteration checks one square farther to the right and one farther square to the left. Thus if the symbol is $n$ squares away from the head at the beginning, it will take $n$ iterations of the machine to reach it. Such a machine is written out below:

$$
   \textbf{begin} \left\{
     \begin{array}{l c c r}
       \text{Any} & N & \textbf{Halt}\\
       \text{None} & Px & \textbf{Halt}\\
     \end{array}
   \right.
 $$
 $$
   \textbf{check-then-left} \left\{
     \begin{array}{l c c r}
       \text{Any} & N & \textbf{Halt}\\
       \text{None} & \textit{Px L} & \textbf{left}\\
     \end{array}
   \right.
$$
$$
   \textbf{left} \left\{
     \begin{array}{l c c r}
       \text{!x} & L & \textbf{left}\\
       \text{x} & \textit{E L} & \textbf{check-then-right}\\
     \end{array}
   \right.
$$
  $$
   \textbf{check-then-right} \left\{
     \begin{array}{l c c r}
       \text{Any} & N & \textbf{Halt}\\
       \text{None} & \textit{Px R} & \textbf{right}\\
     \end{array}
   \right.
 $$
$$
   \textbf{right} \left\{
     \begin{array}{l c c r}
       \text{!x} & R & \textbf{right}\\
       \text{x} & \textit{E R} & \textbf{check-then-left}\\
     \end{array}
   \right.
$$

This machine keeps track of how far it has already checked with two \textit{x} symbols. Each time it checks a new square, it then moves that \textit{x} symbol one farther away from the original head position.

\end{document}
