\documentclass[11pt]{article}
\usepackage{yfonts}
\usepackage{mathrsfs}
\usepackage[font=footnotesize, labelfont={sf,bf}, margin=1cm]{caption}
\usepackage{mathtools}
\usepackage{wrapfig}
\usepackage[letterpaper, portrait, margin=1in]{geometry}
\usepackage{framed}
\usepackage{listings}
\usepackage{color}
\usepackage{float}
\usepackage[utf8]{inputenc}
\usepackage{enumerate}
\usepackage[titletoc,toc,title]{appendix}
\usepackage{amsfonts}
\usepackage{amssymb}
\newcommand{\?}{\stackrel{?}{=}}
%Gummi|065|=)
\title{\textbf{Problem Set 3, Gobbet \#3}}
\author{}
\date{}
\begin{document}

\maketitle

In his proof of Lemma 1, Turing aims to show that for some machine $\mathscr{M}$, if $\mathscr{M}$ will ever print the symbol $S_1$ (which corresponds to a 0) then $Un(\mathscr{M})$ is a provable formula in the Hilbert restricted functional calculus. To define $Un(\mathscr{M})$, Turing first defines several boolean functions which allow us to access all the information of a running machine.

In an effort to improve clarity, we have redefined these functions below:

\vspace{1cm}
{\renewcommand{\arraystretch}{2}
\begin{tabular}{p{3cm} p{8cm}}
%\hline 
$Tape(x, y, S_i)$ & Returns true iff the tape has symbol $S_i$ on square $y$ in complete configuration $x$. \\ 
%\hline 
$ScnSq(x, y)$ & Returns true iff the machine will scan square $y$ in complete configuration $x$. \\ 
%\hline 
$MCon(x, m_i)$ & Returns true iff the m-configuration is $m_i$ in complete configuration $x$.\\ 
%\hline 
$S(a, b)$ & The successor function; returns true if $b$ is the successor of $a$, i.e. $b = a + 1$. \\
%\hline
$SymIn(\xi,\upsilon)$ & Returns the index of the symbol at complete configuration $x$ in square $y$. \\
%\hline
$ScnIn(\xi)$ & Returns the index of the scanned square in complete configuration $x$. \\
%\hline
$MConIn(\xi)$ & Returns the index of the m-configuration in complete configuration $x$. \\
\end{tabular} 
}
\vspace{1cm}

$Inst \ \{ \ m_\iota \ S_\rho \ S_\pi \ L \ m_\phi \ \}$ is an abbreviation for the following formula:
\begin{align}
(\forall x)(\forall y)(\forall x')(\forall y') \bigg\{ & \\
\Big[ S(x, x') \land & S(y', y) \land  \\
& MCon(x, m_\iota) \land ScnSq(x, y) \land Tape(x, y, S_\rho) \Big] \to \\
\Big[ &Tape(x', y, S_\pi) \land ScnSq(x', y') \land MCon(x', m_\phi) \\
&(\forall z)(\forall w) \big[ S(y',z) \lor (Tape(x,z,S_w) \to Tape(x',z,S_w) ) \big] \Big] \bigg\}
\end{align}

For $Inst \ \{ \ m_\iota \ S_\rho \ S_\pi \ R \ m_\phi \ \}$, when the instruction goes right instead of left, we would simply replace the  $S(y', y)$ on line (2) with $S(y, y')$ and $S(y',z)$ on line (5) with $S(z, y')$. 

For $Inst \ \{ \ m_\iota \ S_\rho \ S_\pi \ N \ m_\phi \ \}$, when the instruction does not move, $ScnSq(x',y')$ on line 4 would be replaced by $Scn(x',y)$.

$Des(\mathscr{M})$ is defined as the conjunction of the $Inst$ formulas that correspond to each of $\mathscr{M}$'s instructions after it has been converted to its standard description. 

$Un(\mathscr{M})$

In order to define $Un(\mathscr{M})$, we first need to establish the uniqueness of successors. The following formula, which we will refer to as $Q$, does this:
\begin{align}
(\forall x)(\exists w)(\forall y)(\forall z)\bigg\{S(x, w) \land \\
(S(x, y) \to G(x, y)) \land \\
(S(x, y) \land G(y, z) \to G(x, z)) \land \\
\Big[ G(y, x) \lor \big(G(x, z) \land S(z, y)\big) \lor\\
\big(S(x,z) \land S(y,z)\big) \to \neg S(x, y)) \Big] \bigg\}
\end{align}

$A(\mathscr{M})$

$$\Big[ Q \land ScnSq(\Omega, \Omega) \land (\forall y)(Tape(\Omega, y, S_0) \land MCon(\Omega, m_1) \land Des(\mathscr{M}) \Big ]$$

$Un(\mathscr{M})$
$$(\exists \Omega)A(\mathscr{M}) \to (\exists \alpha)(\exists \beta)Tape(\alpha, \beta, S_1)$$

$CC_\xi$ will be the abbreviation for

%Tape(\Omega^{(\xi)},\Omega'',S_{SymIn(\xi,2)}) \land \\
\begin{align}
Tape(\Omega^{(\xi)},\Omega,S_{SymIn(\xi,0)}) \land Tape(\Omega^{(\xi)},\Omega',S_{SymIn(\xi,1)}) \land \dots \land Tape(\Omega^{(\xi)},\Omega^{(\xi)},S_{SymIn(\xi,\xi)}) \land \\
MCon(\Omega^{(\xi)},m_{MConIn(\xi)}) \land \\
ScnSq(\Omega^{(\xi)},\Omega^{(ScnIn(\xi))}) \land \\
(\forall y)[S(y,\Omega') \lor S(\Omega,y) \lor S(\Omega',y) \lor \dots \lor S(\Omega^{(\xi-1)},y) \lor Tape(\Omega^{(\xi)},y,S_0)]
\end{align}

$S^{(\xi)}$ will be the abbreviation for

$$S(\Omega,\Omega') \land S(\Omega',\Omega'') \land \dots \land S(\Omega^{(\xi-1)},\Omega^{(\xi)})$$

$CF_\xi$ is the abbreviation for

$$A(\mathscr{M}) \land S^{(\xi)} \to CC_\xi$$

We will now show Lemma 1.

Lemma 1: If a zero ($S_1$) is printed on the tape, then the formula $Un(\mathscr{M})$ is provable.

We will first show that $CF_\xi$ is provable for all $\xi$ by mathematical induction.

First, the base case, $CF_0$:

$$A(\mathscr{M}) \land S^{(0)} \to CC_0$$

Since $SymIn(0, n)$ will be 0 for all $n$ at the beginning because we start with a blank tape (as specified in $A(\mathscr{M})$ by the $Tape$ term), then the conjunction of $Tape$ functions in (11) is implied by $A(\mathscr{M})$. Line (12) will reduce to $MCon(\Omega, m_0)$ because $MConIn(0)$ will return 1 as the machine always starts on the first m-configuration, $m_0$. The formula $MCon(\Omega, m_0)$ is contained directly in $A(\mathscr{M})$, and so (12) is also implied. Line (13) reduces to $ScnSq(\Omega, \Omega)$ because $ScnIn(0)$ returns 0 as we start on the 0th square. Since $ScnSq(\Omega, \Omega)$ is also part of $A(\mathscr{M})$ it is also implied by the antecedent. Line (14) will reduce to $(\forall y)[Tape(\Omega,y,S_0)]$, since there are no successor terms when $\xi=0$. This is also implied by the antecedent, since this $A(\mathscr{M})$ also includes the statement of the blank tape.

Therefore, $CF_0$ is provable. We will now show the inductive case, $CC_\xi \to CC_{\xi + 1}$:

$$(A(\mathscr{M}) \land S^{(\xi)} \to CC_\xi) \to (A(\mathscr{M}) \land S^{(\xi + 1)} \to CC_{\xi + 1})$$



\end{document}
