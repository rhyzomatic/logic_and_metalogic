\documentclass[11pt]{article}
\usepackage{yfonts}
\usepackage{mathrsfs}
\usepackage[font=footnotesize, labelfont={sf,bf}, margin=1cm]{caption}
\usepackage{mathtools}
\usepackage{wrapfig}
\usepackage[letterpaper, portrait, margin=1in]{geometry}
\usepackage{framed}
\usepackage{listings}
\usepackage{color}
\usepackage{float}
\usepackage[utf8]{inputenc}
\usepackage{enumerate}
\usepackage[titletoc,toc,title]{appendix}
\usepackage{amsfonts}
\usepackage{amssymb}
\newcommand{\?}{\stackrel{?}{=}}
%Gummi|065|=)
\title{\textbf{Problem Set 3, Gobbet \#3}}
\author{}
\date{}
\begin{document}

\maketitle

In his proof of Lemma 1, Turing aims to show that for some machine $\mathscr{M}$, if $\mathscr{M}$ will ever print the symbol $S_1$ (which corresponds to a 0) then $Un(\mathscr{M})$ is a provable formula in the Hilbert restricted functional calculus. To define $Un(\mathscr{M})$, Turing first defines several boolean functions which allow us to access all the information of a running machine.

In an effort to improve clarity, we have redefined these functions below:

\vspace{1cm}
{\renewcommand{\arraystretch}{2}
\begin{tabular}{p{3cm} p{8cm}}
%\hline 
$Tape(x, y, S_i)$ & Returns true iff the tape has symbol $S_i$ on square $y$ in complete configuration $x$. \\ 
%\hline 
$ScnSq(x, y)$ & Returns true iff the machine will scan square $y$ in complete configuration $x$. \\ 
%\hline 
$MCon(x, m_i)$ & Returns true iff the m-configuration is $m_i$ in complete configuration $x$.\\ 
%\hline 
$S(a, b)$ & The successor function; returns true if $b$ is the successor of $a$, i.e. $b = a + 1$. \\ 
%\hline 
\end{tabular} 
}
\vspace{1cm}

$Inst \ \{ \ m_\iota \ S_\rho \ S_\pi \ L \ m_\phi \ \}$ 

\begin{align}
(\forall x)(\forall y)(\forall x')(\forall y') \bigg\{ & \\
\Big[ S(x, x') \land & S(y', y) \land  \\
& MCon(x, m_\iota) \land ScnSq(x, y) \land Tape(x, y, S_\rho) \Big] \to \\
\Big[ &Tape(x', y, S_\pi) \land ScnSq(x', y') \land MCon(x', m_\phi) \\
&(\forall z)(\forall w) \big[ S(y',z) \lor (Tape(x,z,S_w) \to Tape(x',z,S_w) ) \big] \Big] \bigg\}
\end{align}

For $Inst \ \{ \ m_\iota \ S_\rho \ S_\pi \ R \ m_\phi \ \}$, when the instruction goes right instead of left, we would simply replace the  $S(y', y)$ on line (2) with $S(y, y')$ and $S(y',z)$ on line (5) with $S(z, y')$. 

For $Inst \ \{ \ m_\iota \ S_\rho \ S_\pi \ N \ m_\phi \ \}$, when the instruction does not move, $ScnSq(x',y')$ on line 4 would be replaced by $Scn(x',y)$.

$Des(\mathscr{M})$ is defined as the conjunction of the $Inst$ formulas that correspond to each of $\mathscr{M}$'s instructions after it has been converted to its standard description. 

$Un(\mathscr{M})$

$$(\forall w)(\forall x)(\forall y)(\forall z)\{S(w,x) \land S(w,y) \land S(x,z) \to S(y,z) \}$$
s
\begin{align}
(\forall \Omega) \\
ScnSq(\Omega, \Omega) \\
(\forall y)(Tape(\Omega, y, S_0)) \\
MCon(\Omega, m_1) \\
Desc(\mathscr{M})
\end{align}





\end{document}
